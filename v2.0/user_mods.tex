\section{Installing User Modules}
\label{module installation} \index{modules, installation} \index{installation, modules}
RTXI comes with a set of core system modules. All modules are compiled as Linux shared object libraries that are linked into the core system. This allows RTXI to have minimal overhead and user modules are loaded only as needed. This architecture also allows multiple instantiations of user modules so that elements such as filters and event detectors can be reused on a variety of signals. 

User modules are available on the RTXI website (http://www.rtxi.org/modules) as zipped files or tarballs and also on our GitHub repository (https://github.com/RTXI). Each module usually consists of a single class header file (*.h), class implementation file (*.cpp), a Makefile that informs the GCC compiler, and a directory for legacy modules used in previous versions of RTXI. For simplicity, we recommend that user modules be stored together in a single directory (such as \texttt{\$HOME/modules}). To extract a module compressed as a tarball, in this case the plugin template, directory and compile the module:
\begin{example}
\$ tar xvf plugin\_template.tar.gz\\
\$ cd plugin-template\\
\$ sudo make install
\end{example}

If you have Git installed, you can also download and install modules by running:
\begin{example}
\$ git clone https://github.com/rtxi/plugin-template.git\\
\$ cd plugin-template\\
\$ sudo make install\\
\end{example}

This process will create an RTXI shared object library (*.so extension), which will then be copied to \texttt{/usr/local/lib/rtxi}, where RTXI will initially look for them. User modules must be recompiled if any changes are made to their sources after installation. Note that when reinstalling modules, the corresponding *.so files are overwritten, so always make sure that different modules have unique names. Instructions for writing custom user modules are given in Chapter \ref{custommodules}.