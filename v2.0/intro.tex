\chapter{Introduction}

\section{About RTXI}

The Real-Time eXperiment Interface (RTXI) is a collaborative open-source software development project aimed at producing a real-time Linux based software system for hard real-time data acquisition and control applications in biological research. RTXI merges three previous systems for closed-loop biological experiments: RTLab \cite{Culianu:2001fk,Culianu:2002uq}, Real-time Linux Dynamic Clamp (RTLDC) \cite{Dorval:2001p940}, and Model Reference Current Injection (MRCI) \cite{Butera:2001p910,Raikov:2004p1166}. RTLDC and MRCI focus on implementing dynamic clamp, an experimental technique in cardiac and neural electrophysiology that is used to simulate ionic membrane currents. RTXI combines the features of all three predecessor platforms into a more general platform for real-time closed-loop experimental protocols. Using real-time control, scientists can quantify biological function via perturbations that change according to closed-loop analysis of measured system variables, rather than being restricted to measuring responses to pre-determined stimuli. Real-time control applications are abundant throughout biological research, including, for example, dynamic probing of ion-channel function, control of cardiac arrhythmia dynamics, and control of deep-brain stimulation patterns. There is a wide range of biological research endeavors for which real-time control can offer insight that cannot be obtained with traditional methods. 

RTXI is based on Linux, which is extended with Xenomai \cite{Xenomai:vn} to provide a hard real-time platform with the comprehensive Linux desktop environment \footnote{Legacy support is provided for the Real-time Applications Interface (RTAI) \cite{RTAI:vn} and Linux Control and Measurement Device Interface (COMEDI) \cite{COMEDI:uq}}. Data acquisition and analog/digital interfaces to other hardware are implemented in real-time using the Analogy real-time driver interface, which provides support to a variety of commercial multifunction data acquisition cards. Experimental protocols and other real-time algorithms are implemented within a modular framework that  allows users to easily reuse existing code and construct complex protocols. Users can also take advantage of previously written code or other \cpp libraries to add functionality to their modules. As such, RTXI is a generic real-time platform with potential applications beyond dynamic clamp.\

RTXI is released under a combination of the GPL and LGPL licenses. The core RTXI code is covered by a GPL license but user modules distributed as binary libraries are covered under LPGL and their source code may be available at the discretion of the original authors. All documentation is released under the GNU Free Documentation License.  \index{MRCI} \index{RTLDC} \index{RTLab} \index{licensing}

\bigskip

\hrule
\bigskip
\attention
If your use of RTXI leads to scientific publication, we request that you cite RTXI in your paper with text such as: ``Experiments were performed using the Real-Time eXperiment Interface (RTXI; www.rtxi.org)."\bigskip
\hrule