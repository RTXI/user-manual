\section{Hardware Requirements}

\index{hardware requirements} \index{compatibility, hardware}
RTXI is designed to run on a standard personal desktop computer. \index{real-time performance}Computers with uniprocessors, multi-core processors, and multiple CPUs with and without hyperthreading are supported by Linux and RTXI, though typically more stable systems are realized with Intel rather than AMD processors. For multi-core computers, Xenomai will need to be manually configured during installation. There are very few CPU and motherboard combinations that are not supported, and those tend to be advanced motherboards with features that are not compatible with Xenomai. For example, some that use integrated graphics chips use hardware techniques to speed up computation that are also not compatible. An external video card is recommended and NVIDIA cards tend to have better Linux support than ATI/Radeon cards. This can be important since the greatest overhead in RTXI is related to data visualization in the oscilloscope. For newer graphics cards, you may need to manually install the Linux drivers, which are usually available on the manufacturer's website. Some systems may also include BIOS level or hardware interrupts that are not captured by Xenomai or advanced power management features. Sometimes, BIOS allows these features to be disabled.

The real-time Linux kernel has extremely low latencies and little software overhead. RTXI is also designed so that the minimum number of dynamic modules can be loaded at any time. RTXI has been successfully tested on computers with Pentium III processors up to 6-core Intel Xeon processors. While the processor speed allows RTXI to complete more computations within a single real-time cycle, the amount of RAM and the amount of video memory have a significant impact on the stability and speed of the system. Users should also consider high speed hard drives, large cache sizes, and high speed bus interfaces. If you are purchasing an off-the-shelf desktop computer system and plan to add a DAQ card, be sure that your power supply is powerful enough to handle the extra load. At least a 450W power supply is recommended. 