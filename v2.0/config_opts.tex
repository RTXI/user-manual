\newpage
\section{RTXI Configuration Options}
\index{RTXI, configuration}
RTXI can be manually configured with other options. For example, you may want to run RTXI using the RTAI real-time interface rather than Xenomi or in non-real-time mode using the POSIX interface for debugging purposes. You may also direct RTXI to libraries/packages in non-standard locations. The full configuration options are below:

\begin{maxipage}
\begin{example}
Usage: ./configure [OPTION]... [VAR=VALUE]...To assign environment variables (e.g., CC, CFLAGS...), specify them as VAR=VALUE.  See below for descriptions of some of the useful variables. Defaults for the options are specified in brackets.

Configuration:\\ \vspace{.4cm}
\begin{tabular}{ll}
-h, --help              &     display this help and exit\\
--help=short           &    display options specific to this package\\
--help=recursive    &   display the short help of all the included packages\\
-V, --version           &    display version information and exit\\
-q, --quiet, --silent   &  do not print `checking...' messages\\
--cache-file=FILE   &  cache test results in FILE [disabled]\\
-C, --config-cache    &    alias for `--cache-file=config.cache'\\
-n, --no-create       &    do not create output files\\
--srcdir=DIR     &     find the sources in DIR [configure dir or `..']\\
\end{tabular}

Installation directories:\\ \vspace{.4cm}
\begin{tabular}{ll}
--prefix=PREFIX       &    install architecture-independent files in PREFIX \\
& [/usr/local]\\
--exec-prefix=EPREFIX   &  install architecture-dependent files in EPREFIX [PREFIX]
\end{tabular}

By default, `make install' will install all the files in
`/usr/local/bin', `/usr/local/lib' etc.  You can specify
an installation prefix other than `/usr/local' using `--prefix',
for instance `--prefix=\$HOME'.

For better control, use the options below.\\ \vspace{.4cm}

Fine tuning of the installation directories:\\ \vspace{.4cm}
\begin{tabular}{ll}
--bindir=DIR         &     user executables [EPREFIX/bin]\\
--sbindir=DIR       &      system admin executables [EPREFIX/sbin]\\
--libexecdir=DIR    &      program executables [EPREFIX/libexec]\\
--sysconfdir=DIR      &    read-only single-machine data [PREFIX/etc]\\
--sharedstatedir=DIR    &  modifiable architecture-independent data [PREFIX/com]\\
--localstatedir=DIR    &   modifiable single-machine data [PREFIX/var]\\
--libdir=DIR     &         object code libraries [EPREFIX/lib]\\
--includedir=DIR     &     C header files [PREFIX/include]\\
--oldincludedir=DIR  &     C header files for non-gcc [/usr/include]\\
--datarootdir=DIR    &     read-only arch.-independent data root [PREFIX/share]\\
--datadir=DIR       &      read-only architecture-independent data [DATAROOTDIR]\\
\end{tabular}

\end{example}
\end{maxipage}

\begin{maxipage}
\begin{example}
\begin{tabular}{ll}
--infodir=DIR        &     info documentation [DATAROOTDIR/info]\\
--localedir=DIR   &        locale-dependent data [DATAROOTDIR/locale]\\
--mandir=DIR     &         man documentation [DATAROOTDIR/man]\\
--docdir=DIR      &        documentation root [DATAROOTDIR/doc/rtxi]\\
--htmldir=DIR     &        html documentation [DOCDIR]\\
--dvidir=DIR       &       dvi documentation [DOCDIR]\\
--pdfdir=DIR      &        pdf documentation [DOCDIR]\\
--psdir=DIR      &         ps documentation [DOCDIR]
\end{tabular}

Program names:\\ \vspace{.4cm}
\begin{tabular}{ll}
--program-prefix=PREFIX    &          prepend PREFIX to installed program names\\
--program-suffix=SUFFIX      &        append SUFFIX to installed program names\\
--program-transform-name=PROGRAM   &  run sed PROGRAM on installed program names
\end{tabular}

X features:\\ \vspace{.4cm}
\begin{tabular}{ll}
--x-includes=DIR   &   X include files are in DIR\\
--x-libraries=DIR  &   X library files are in DIR
\end{tabular}

System types:\\ \vspace{.4cm}
\begin{tabular}{ll}
--build=BUILD    &   configure for building on BUILD [guessed]\\
--host=HOST     &    cross-compile to build programs to run on HOST [BUILD]
\end{tabular}

Optional Features:\\ \vspace{.4cm}
\begin{tabular}{ll}
--disable-option-checking &   ignore unrecognized --enable/--with options\\
--disable-FEATURE     &  do not include FEATURE (same as --enable-FEATURE=no)\\
--enable-FEATURE[=ARG]  &  include FEATURE [ARG=yes]\\
--enable-shared[=PKGS]   & build shared libraries [default=yes]\\
--enable-static[=PKGS]  &  build static libraries [default=yes]\\
--enable-fast-install[=PKGS]  & optimize for fast installation [default=yes]\\
--disable-dependency-tracking  &  speeds up one-time build\\
--enable-dependency-tracking   &  do not reject slow dependency extractors\\
--disable-libtool-lock  &  avoid locking (might break parallel builds)\\
--enable-rtai      &   build the Xenomai interface\\
--enable-posix      &     build the POSIX non-RT interface\\
--enable-debug       &    turn on debugging\\
--enable-comedi    &      build the comedi driver\\
--enable-ni        &      build the ni driver
\end{tabular}

Optional Packages:\\ \vspace{.4cm}
\begin{tabular}{ll}
--with-PACKAGE[=ARG]    &  use PACKAGE [ARG=yes]\\
--without-PACKAGE      &   do not use PACKAGE (same as --with-PACKAGE=no)\\
--with-cppunit-prefix=PFX   &  Prefix where CppUnit is installed (optional)\\
--with-cppunit-exec-prefix=PFX   & Exec prefix where CppUnit is installed (optional)\\
--with-pic       &         try to use only PIC/non-PIC objects [default=use both]\\
--with-gnu-ld       &      assume the C compiler uses GNU ld [default=no]\\
--with-x            &      use the X Window System\\
\end{tabular}
\end{example}
\end{maxipage}

\begin{maxipage}
\begin{example}
\begin{tabular}{ll}

--with-Qt-dir=DIR      &   DIR is equal to \$QTDIR if you have followed the \\
& installation instructions of Trolltech. Header files \\
& are in DIR/include, binary utilities are in DIR/bin. \\
& The library is in DIR/lib, unless --with-Qt-lib-dir \\
& is also set.\\
--with-Qt-include-dir=DIR  & Qt header files are in DIR\\
--with-Qt-bin-dir=DIR  &   Qt utilities such as moc and uic are in DIR\\
--with-Qt-lib-dir=DIR   &  The Qt library is in DIR\\
--with-Qt-lib=LIB      &   Use -lLIB to link with the Qt library\\
--with-rtai-config=FILE  & location of the rtai-config program
\end{tabular}

Some influential environment variables:\\ \vspace{.4cm}
\begin{tabular}{ll}
CC      &      C compiler command\\
CFLAGS    &    C compiler flags\\
LDFLAGS   &    linker flags, e.g. -L<lib dir> if you have libraries in a nonstandard\\
&  directory <lib dir>\\
LIBS   &       libraries to pass to the linker, e.g. -l<library>\\
CPPFLAGS  &    C/C++/Objective C preprocessor flags, e.g. -I<include dir> if you have\\
& headers in a nonstandard directory <include dir>\\
CPP       &    C preprocessor\\
CXX       &    C++ compiler command\\
CXXFLAGS   &   C++ compiler flags\\
CXXCPP    &    C++ preprocessor\\
XMKMF     &    Path to xmkmf, Makefile generator for X Window System
\end{tabular}

Use these variables to override the choices made by `configure' or to help it to find libraries and programs with nonstandard names/locations.

\end{example}
\end{maxipage}